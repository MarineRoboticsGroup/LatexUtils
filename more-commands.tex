% dehann's latex arguments and commands
% started 11/26/2013 at 8:08PM on Monday

\usepackage{accents}

% \DeclareMathOperator{\argmax}{argmax}
% \DeclareMathOperator{\argmin}{argmin}
\DeclareMathOperator*{\argmin}{\arg\!\min}
\DeclareMathOperator*{\argmax}{\arg\!\max}

\DeclareMathOperator*{\solve}{solve}

% Primitives

\newcommand{\cosb}[1]{\cos{\left( #1 \right)}}
\newcommand{\sinb}[1]{\sin{\left( #1 \right)}}

% Vector and matrix commands
\DeclareMathOperator{\Tr}{Trace}

\newcommand{\B}[1]{\begin{bmatrix} #1 \end{bmatrix}}
\newcommand{\skewx}[1]{\left[ #1_{\times} \right]}

\newcommand{\dgr}{^{\circ}}

\newcommand{\vb}[1]{\mathbf #1}

\newcommand{\vB}[1]{ \begin{bmatrix} #1_x \\ #1_y \\ #1_z \end{bmatrix} }
\newcommand{\vBT}[1]{ \begin{bmatrix} #1_x & #1_y & #1_z \end{bmatrix} }
\newcommand{\vBTn}[1]{ \begin{bmatrix} #1_1 & #1_2 & #1_3 \end{bmatrix} }

\newcommand{\uv}[1]{ \hat{\vb{#1}} }

\DeclareMathOperator{\st}{s.t.}

\newcommand{\skewsym}[1]{\begin{bmatrix} 0 & -#1_z & #1_y \\
					  #1_z & 0 & -#1_x \\
					  -#1_y & #1_x & 0\end{bmatrix}}

\newcommand{\bmat}[1]{\begin{bmatrix} #1 \end{bmatrix} }

% Quaternion commands
\newcommand{\q}[1]{\accentset{\circ}{\mathbf{#1}}}
\newcommand{\qv}[1]{\begin{bmatrix} #1_0 \\ #1_1 \\ #1_2 \\ #1_3 \end{bmatrix}}
\newcommand{\qvT}[1]{\begin{bmatrix} #1_0 & #1_1 & #1_2 & #1_3 \end{bmatrix}}
\newcommand{\qprodMe}[4]{ \begin{bmatrix} #1 & -#2 & -#3 & -#4 \\
					#2 & #1 & -#4 & #3 \\
					#3 & #4 & #1 & -#2 \\
					#4 & -#3 & #2 & #1 \\\end{bmatrix} }
\newcommand{\qRprodMeConj}[4]{ \begin{bmatrix} #1 & #2 & #3 & #4 \\
						-#2 & #1 & -#4 & #3 \\
						-#3 & #4 & #1 & -#2 \\
						-#4 & -#3 & #2 & #1 \\ \end{bmatrix} }
\newcommand{\qprodM}[1]{\qprodMe{#1_0}{#1_1}{#1_2}{#1_3}}
\newcommand{\qprod}[2]{ \qprodM{#1} \qv{#2}}

\newcommand{\qRot}[3]{\q{#1} \q{#2}^{#3} \q{#1}^*}




\newcommand{\fnc}[2]{#1 \left( #2 \right)}
\newcommand{\f}[2]{#1 \left( #2 \right)}
\newcommand{\innp}[2]{\langle \, #1, \, #2  \, \rangle}
\newcommand{\inner}[2]{\langle \, #1, \, #2  \, \rangle}

% \newcommand{\fnc}[3]{#1 \left( #2, #3 \right)}

\newcommand{\bel}[1]{ \left[ \, #1 \, \right]}
\newcommand{\cbel}[2]{ \left[ \, #1 \, | \, #2 \, \right]}


% probability

\newcommand{\E}[2]{\mathbb{E}_{#1} \left[ #2 \right]}


\newcommand{\dKL}[2]{\fnc{D_{KL}}{#1 \| #2}}

% figure commands

\newcommand{\fref}[1]{Fig.~\ref{#1}}
\newcommand{\Fref}[1]{Fig.~\ref{#1}}





% names

\newcommand{\MATLAB}{\texttt{MATLAB}${}^\text{\textregistered}$}



\newcommand{\eqnref}[1]{\textnormal{(}\ref{#1}\text{)}}
%To print R, the real number symbol
\newcommand{\R}{\mathbb{R}}
%To print Q, the rational number symbol
\newcommand{\Q}{\mathbb{Q}}
%To print Z, the integer symbol
\newcommand{\Z}{\mathbb{Z}}
%To print N, the natural number symbol
\newcommand{\N}{\mathbb{N}}
%To print H, the upper half-plane symbol
\newcommand{\Hp}{\mathbb{H}}
%To print C, the complex number symbol
\newcommand{\C}{\mathbb{C}}
%To get the tensor space T symbol
\newcommand{\T}{\mathfrak{J}}
%To get the finite field symbol F
\newcommand{\F}{\mathbb{F}}
\newcommand{\Ha}{\mathbb{H}}
\newcommand{\PR}{\mathbb{PR}}
\newcommand{\so}{\mathfrak{so}}
\newcommand{\gl}{\mathfrak{gl}}

\newcommand{\Nor}{\mathcal{N}}



%To get some commonly used mathematical operators

%Differential Topology

%The codimension operator for manifolds, codim
\DeclareMathOperator{\codim}{codim}
%The image operator
\DeclareMathOperator{\image}{Image}
%The volume operator, vol
\DeclareMathOperator{\vol}{vol}
%The topological interior operator, Int
\DeclareMathOperator{\Int}{Int}
%The null space operator, N
\DeclareMathOperator{\Null}{N}
%Standard sign operator, sign
\DeclareMathOperator{\sign}{sign}
%The graph operator; used to denote the graph of a parametrization considered as a manifold
\DeclareMathOperator{\Graph}{graph}
%The topological index operator
\DeclareMathOperator{\ind}{ind}
%Standard divergence operator
\DeclareMathOperator{\divergence}{div}
%Standard curl operator
\DeclareMathOperator{\curl}{curl}
%The alternating tensor operator, used for de Rham cohomology in differential topology
\DeclareMathOperator{\Alt}{Alt}

%Abstract algebra

%The automorphism group of a given group.
\DeclareMathOperator{\Aut}{Aut}
%The inner automorphism group of a given group.
\DeclareMathOperator{\Inn}{Inn}
%The outer automorphism group
\DeclareMathOperator{\Out}{Out}
%The Galois group of an extension.
\DeclareMathOperator{\Gal}{Gal}
%The set of embeddings of E into a fixed algebraic closure of F
\DeclareMathOperator{\Emb}{Emb}
%Used to denote the set of Sylow p-subgroups of a group.
\DeclareMathOperator{\Syl}{Syl}
%Used to get the characteristic subgroup operator, char.
\DeclareMathOperator{\charac}{char}
%Used to get the least common multiple abbreviation, lcm.
\DeclareMathOperator{\lcm}{lcm}
%Used to get the degree operator, for the degree of a polynomial
% \DeclareMathOperator{\degree}{degree}
%Used to get the characteristic of a field
\DeclareMathOperator{\fch}{ch}
%The radical of an ideal
\DeclareMathOperator{\rad}{rad}

\DeclareMathOperator{\Hom}{Hom}

%Used to represent the special linear group
\DeclareMathOperator{\SL}{SL}
%Used to represent the general linear group
\DeclareMathOperator{\GL}{GL}
%Used to represent the projective special linear group of matrices
\DeclareMathOperator{\PSsequentialL}{PSL}
%To get Span
\DeclareMathOperator{\Span}{Span}
%To get id
\DeclareMathOperator{\id}{id}

\DeclareMathOperator{\Norm}{Norm}
